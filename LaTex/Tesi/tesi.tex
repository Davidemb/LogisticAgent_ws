%\documentclass[11pt, twoside, titlepage, a4paper, openright]{report}
\documentclass[12pt, titlepage, a4paper]{book}
\usepackage{graphicx}
\usepackage[english]{babel}
\usepackage[utf8]{inputenc}
\usepackage{algpseudocode}
\usepackage{amsthm}
\usepackage{algorithm}
\usepackage{rotating}
\usepackage{enumerate}
%\usepackage[Lenny]{fncychap}
\usepackage{hyperref}
\hypersetup{
    colorlinks=true,       % false: boxed links; true: colored links
    linkcolor=black,%red,          % color of internal links
    citecolor=black,%green,        % color of links to bibliography
    filecolor=black,%magenta,      % color of file links
    urlcolor=black%cyan           % color of external links
}
% \usepackage [a4paper,left=2.5cm,bottom=2.5cm,right=2.5cm,top=3cm]{geometry}
\usepackage{frontespizio}
\linespread{1.2}
\newtheorem{mydef}{Definition} % definizioni <-----
\usepackage{listings}
\lstset{
	language=C++,
	basicstyle=\small\ttfamily, 
	numbers=left,
	numberstyle=\tiny,
	frame=tb,
	columns=fullflexible,
	showstringspaces=false
}
\usepackage{multicol}
% \usepackage[style=numeric]{biblatex}
% \usepackage{multicol}

%\nofiles 
%\fontoptionnormal 
\usepackage{mathtools}

\frenchspacing

\newcommand{\srst}[0]{SR:ST\space}
\newcommand{\gsp}[0]{GSP1:N\space}
\newcommand{\sps}[0]{SPS1:N\space}
\newcommand{\mrs}[0]{MRS\space}
\begin{document}


\begin{frontespizio}
\Universita {Verona}
\Dipartimento {Informatica}
\Corso {Ingegneria e Scienze Informatiche}
\Annoaccademico {2017-2018}
\Titoletto {Tesi di Laurea Magistrale}
\Titolo { \Large {Multi-Robot Task Allocation for logistic applications} }
\Candidato [vr414572]{Davide Zorzi}
\Relatore {Alessandro Farinelli}
\Rientro {1.5cm}
\NCandidato {Candidato}
\end{frontespizio}

%\clearpage\null\thispagestyle{plain}\clearpage
%\newpage

%\preparefrontpage

% \bibliographystyle{unsrt}
% \bibliography{sample}

\tableofcontents

\newpage

\paragraph{Abstract}
\begin*{}
\newline
\newline
Robotics technology has recently matured sufficiently to deploy autonomous
robotic systems for daily use in several applications: from disaster response
to environmental monitoring and logistics.
In this project we present and evaluate the main difference of central allocator task coordinator. 
In this application we address off-line coordination, by casting the Multi-Robot
logistics problem as a task assignment problem and proposing three solution 
techniques: Single robot Single task (\srst), which is 
a baseline greedy approach,  Greedy Set Partition Strategy Single robot Multiple task 
(\gsp) and Set Partition Strategy Single robot Multiple task (\sps),
which are based on merging task to minimize the task completion time.
\\
We evaluate the performance of our system in a realistic simulation enviroment
(build with ROS and stage). In particular, in the simulated enviroment we compare
our task assignment approaches with the baseline greedy approch. Results show that fully
exploiting the capacities of robots is key to optimze system performance and that the 
\gsp achieves similar performance to \sps while being able to scale up to many tasks.
\newline
\newline
\textbf{Keywords:} Multi-Robot Task Allocation, logistic applications, Multi-Robot
systems, coordination, task assignment

% TODO: primi risultati e considierazioni

\end*{}




\chapter{Introduction}\label{chap:intro}

    \begin{frame}[fragile]{Industrial Logistics}
        The {\bf industrial logistics} is the process of {\bf planning}, {\bf organization}
        and {\bf control} of all the activities of handling and {\bf storage} of goods, which, starting
        from the suppliers and reaching up to the end user, guarantee an adequate
        level of {\bf service} to the customer consistent with the {\bf costs} to it associated

        \begin{figure}[hbt]
            \centering
            \includegraphics[width=\textwidth]{img/ind4.png}
        \end{figure}
    \end{frame}

    \begin{frame}[fragile]{Multi-Robot Systems for logistic applications}

        \begin{figure}[hbt]
            \centering
            \includegraphics[width=\textwidth]{img/kiva.png}
        \end{figure}
        
        \begin{center}
        {\bf Kiva} warehouse-management system.
        \end{center}
    \end{frame}

    \begin{frame}[fragile]{Thesis contribution}
        \begin{center}
        The contribution of this thesis:
        \end{center}
        \begin{columns}
            \begin{column}{.7\textwidth}
           
            \begin{itemize}
            \item extension of  \texttt{ROS}  package
            \item proposing three tequnique:
            \begin{enumerate}
                \item Single robot : Single task (SR:ST) 
                \item Set Partition Strategy - Single robot : Multiple task (SPS1:N)
                \item Greedy Set Partition Strategy - Single robot : Multiple task (GSP1:N)
            \end{enumerate}
                \item real scenario: Computer Engineering for Industry 4.0 Laboratory (ICE Lab) 
            \end{itemize}
            \end{column}
            \begin{column}{.4\textwidth}
            \begin{figure}
                \subfloat{\includegraphics[scale=0.45]{img/ros}}\qquad
                \subfloat{\includegraphics[scale=0.45]{img/ice}}
            \end{figure}
            \end{column}
        \end{columns}
    \end{frame}


    \begin{frame}[fragile]{ICE Laboratory}
        \begin{figure}[hbt]
            \centering
            \includegraphics[width=\textwidth]{img/model1}
        \end{figure}
    \end{frame}

    \begin{frame}[fragile]{ICE Laboratory for logistic application}
        \begin{figure}[hbt]
            \centering
            \includegraphics[width=\textwidth]{img/labgrafo}
        \end{figure}

        {\color{red}{$\bullet$}}  Loading bay
        \\
        {\color{blue}{$\bullet$}}  Unloading bays
        \\
        {\color{black}{$\circ$}}  Vertices
    \end{frame}

    \begin{frame}[fragile]{Problem formalization}
        Given a set of tasks $\mathcal{T}$ it defines intrinsically a set of orders $O$.
        One order perform a subset $S$ of $\mathcal{T}$, $S\subseteq\mathcal{T}$.

        $S = \{T_1,\cdots,T_k\}$ for each element we {\bf combine} their paths $P$ to form
        a single path $\pi = \{v_1,\cdots,v_i\}$.
        \begin{figure}
           
                \includegraphics[scale=0.18]{img/p1p2_cut.png}
                \includegraphics[scale=0.18]{img/p3_cut.png}

        \end{figure}
    \end{frame}

    \begin{frame}[fragile]{Problem formalization 2}
       We {\bf maximize} the total demand ($d_S$) we calculate the sum the single demand for each element in the subset.
\[d_S =demand(T_1) + \cdots + demand(T_k)\]
The heuristic function $v(\cdot)$ which
can be defined for any task $T$ or subset $S$:
\[ v(S) = \frac{f(\pi)}{d_S}\]

For compute the best partition of tasks the heuristic is based on the concept of {\bf loss} $L$,
which can be defined for any pair of subset $S_i$, $S_j$ as:
\[L(S_i,S_j) = v(\{ S_i \cup S_j\}) - v(S_i) - v(S_j)\]
where $v(S)$ is value of the characteristic function $v(\cdot)$ for subset $S$.

We want {\bf minimize} the cost of the {\bf loss}, that can be defined as:
\[L(S_i,S_j) < 0 \]
if the loss $L$ is less than 0 then we allocate the pair and delete the element which
form the subset.
    \end{frame}

    \begin{frame}[fragile]{Single robot : Single task (SR:ST)}
        This method is a {\bf baseline} for our logistic scenario.

        \begin{figure}[hbt]
            \centering
            \includegraphics[width=\textwidth]{img/cycle1.png}
        \end{figure}
        The important constraint of this approach is to consider only {\bf one task} allocated for 
{\bf one robot} at time.
    \end{frame}

    \begin{frame}[fragile]{Set Partition Strategy - Single robot : Multiple task (SPS1:N)}
         \begin{columns}
            \begin{column}{.7\textwidth}
                \begin{center}
                    \begin{tabular}{|c|r|c|} \hline
                    \textbf{iteration} & \textbf{partition size} & \textbf{partition} \\ \hline
                    1    & 1    & \{\{a, b, c, d\}\}   \\
                    2    & 2    & \{\{a, b, c\}, \{d\}\}   \\
                    3    & 2    & \{\{a, b, d\}, \{c\}\}   \\
                    4    & 2    & \{\{a, b\}, \{c, d\}\}   \\
                    5    & 3    & \{\{a, b\}, \{c\}, \{d\}\}   \\
                    6    & 2    & \{\{a, c, d\}, \{b\}\}   \\
                    7    & 2    & \{\{a, c\}, \{b, d\}\}   \\
                    8    & 3    & \{\{a, c\}, \{b\},\{d\}\}   \\
                    9    & 2    & \{\{a, d\}, \{b, c\}\}   \\
                    10   & 2    & \{\{a\}, \{b, c, d\}\}   \\
                    11   & 3    & \{\{a\}, \{b, c\}, \{d\}\}   \\
                    12   & 3    & \{\{a, d\}, \{b\}, \{c\}\}   \\
                    13   & 3    & \{\{a\}, \{b, d\}, \{c\}\}   \\
                    14   & 3    & \{\{a\}, \{b\}, \{c, d\}\}   \\
                    15   & 4    & \{\{a\}, \{b\}, \{c\},\{d\}\}   \\ \hline       
                    \end{tabular}
                  \end{center}
            \end{column}
            \begin{column}{.4\textwidth}
            \begin{figure}
                \includegraphics[width=\textwidth]{img/exp}
            \end{figure}
            \end{column}
        \end{columns}
    \end{frame}

    \begin{frame}[fragile]{Greedy Set Partition Strategy - Single robot : Multiple task (GSP1:N)}
        The main concept of this approach is composing tasks using Greedy {\bf Coalition Formation} strategy.
        \begin{figure}[hbt]
            \centering
            \includegraphics[width=\textwidth]{img/CF.png}
        \end{figure}
        The horizontal line represents a cut during execution it defines the coalition structure.
    \end{frame}

    \begin{frame}[fragile]{Example}
        Given a set of tasks $\mathcal{T}= \{  \{T_0\}, \{T_1\}, \cdots, \{T_8\} \}$ defined like:

        ${T_i=(item, demand, unloading\_bay)}$.
        
        The agents have the same capacity $C_{0,1,2,3} = 4$.
                    \begin{center}
                      % \centering
                      \begin{tabular}{|c|c|c|c|} \hline
                        \textbf{task} & \textbf{item} & \textbf{demand} & \textbf{unloading bay} \\ \hline
                        0    & A    & 1      & 0             \\
                        1    & B    & 2      & 1             \\
                        2    & C    & 3      & 2             \\
                        3    & A    & 1      & 0             \\
                        4    & B    & 2      & 1             \\
                        5    & C    & 3      & 2             \\
                        6    & A    & 1      & 0             \\
                        7    & B    & 2      & 1             \\
                        8    & C    & 3      & 2             \\ \hline       
                      \end{tabular} 
                    \end{center}
                    Often in the logistic environments robots are all equal. 
            
    \end{frame}

    \begin{frame}[fragile]{Example 2}
        Result SPS:
            \begin{center}
              \begin{tabular}{|c|c|c|c|} \hline
              \textbf{task} & \textbf{item} & \textbf{demand} & \textbf{unloading bay} \\ \hline
              \{4,7\}    & B    & 4     & 1             \\
              \{0,1,3\}  & \{A,B\}& 4    & \{0,1\}             \\
              \{2,6\}    & \{C,A\}    & 4  & \{0,2\}             \\
              5    & C    & 3      & 2             \\
              8    & C    & 3      & 2             \\ \hline       
              \end{tabular}
              
            \end{center}
   Result GSP:
            \begin{center}
              \begin{tabular}{|c|c|c|c|} \hline
              \textbf{task} & \textbf{item} & \textbf{demand} & \textbf{unloading bay} \\ \hline
              \{3,2\}    & \{A,C\}    & 4     & \{0,2\}             \\
              \{0,1\}    & \{A,B\}    & 3     & \{0,1\}             \\
              \{6,4\}    & \{A,B\}    & 3     & \{0,1\}             \\
              5    & C    & 3      & 2             \\
              8    & C    & 3      & 2             \\        
              7    & B    & 2      & 1             \\\hline
              \end{tabular}
              
            \end{center}
       
    \end{frame}

    \begin{frame}[fragile]{Video}
        \movie[height=6.5cm,width=11.5cm,loop]{}{video/video.mp4}
    \end{frame}

    \begin{frame}[fragile]{ROS package Logistic\_sim}
        \begin{figure}[hbt]
            \centering
            \includegraphics[scale=0.12]{img/rosgraph}
        \end{figure}
    \end{frame}

    \begin{frame}[fragile]{Empirical Results}
        \resizebox{\linewidth}{4.3cm}{
        \begin{tabular}{|c|c|c|c|c|c|} \hline
            {\bf Configuration} & {\bf Algorithm} & {\bf $ \overline{Time}$} & {\bf $\overline{Interference}$} & {\bf $\overline{Distance}$} & {\bf $\bar{\sigma}(Distance)$}         \\ \hline
            6/-/2               & \srst           & 218.32$[\pm 6.19]$        & 63.45   & 3747.90 &  87.8\\ \hline 
            6/3/2               & \gsp            & 194.52$[\pm 6.42]$        & 49.65    & 3401.15 & 251.37  \\ 
                                & \sps            & 177.00$[\pm 1.99]$        & 49.34    & 3132.5 & 0  \\ \hline
            6/5/2               & \gsp            & 142.08$[\pm 1.39]$       &   42.2        & 2714.25      &  206.43  \\
                                & \sps            & 138.98$[\pm 2.41]$       &  39.38   &  2601.25  &  156.47   \\ \hline
            6/-/4               & \srst           & 124.52$[\pm 3.12]$        & 42    & 2194.75 &  114.2 \\ \hline
            6/3/4               & \gsp            & 117.44$[\pm 1.85]$        & 35.75   & 1769 & 43.83  \\ 
                                & \sps            & 115.28$[\pm 4.10]$        & 33.5   & 1702.5 & 23.67  \\ \hline
            6/5/4               & \gsp            &  93.4$[\pm 1.01]$         & 29     & 1688.5 &  34.5   \\
                                & \sps            &  91.8$[\pm 2.14]$        & 30.75  & 1546.5 & 35.8  \\ \hline
            9/-/2               & \srst           & 292.24$[\pm 3.06]$      &  85.5  & 5201.5 & 34.76\\ \hline
            9/3/2              & \gsp            & 265.72$[\pm 2.64]$      & 71.5   & 4491.5  & 0  \\ 
                               & \sps            & 240.74$[\pm 10.42]$        & 75.5    & 4232.5 & 310.43 \\ \hline
             9/5/2             & \gsp          & 232.84$[\pm 4.71]$      &  68.85  &  4041.25  &  236 \\
                               & \sps           &168.34$[\pm 2.03]$       &   50.5   & 3132.5 &  0  \\ \hline 
            9/-/4               & \srst           & 178.55$[\pm 4.23]$     & 52  & 2755.75 & 135.8 \\ \hline
            9/3/4               & \gsp            & 152.55$[\pm 2.87]$     & 46.75 & 2200 & 113.4 \\ 
                               & \sps            & 134.23$[\pm 3.25]$     &  40.63   & 2182.5 & 27 \\ \hline
            9/5/4              & \gsp           &134.23$[\pm 3.26]$        & 40.6    & 2098.3     &   93.45    \\
                               & \sps           &93.05$[\pm 5.15]$         & 32.25    & 1530.25 &   0 \\ \hline    
            21/-/2             & \srst           & 629.10$[\pm 8.84]$        & 154.6   &11773.5 &  229.75 \\ \hline
            21/3/2              & \gsp            & 561.93$[\pm 8.00]$     &  134.3  &10133.16 &  201.2   \\ 
            21/5/2              & \gsp            & 497.45$[\pm 6.15]$      & 126  & 9079 & 210.4  \\ \hline
            21/-/4              & \srst           & 402.12$[\pm 5.06]$     & 132.25  & 6232.35 &  295.1 \\ \hline
            21/3/4              & \gsp            & 343.23$[\pm 6.10]$ & 98.23 & 5231.25 & 342.2   \\ 
            21/5/4              & \gsp            & 294.40$[\pm 7.60]$ & 77.63 & 4683.25 & 367.5  \\ \hline
        \end{tabular}}
    \end{frame}

   

    \begin{frame}[fragile]{Conclusions and Future Work}
        In conclusion: 
        \begin{itemize}
            \item Good behavior of GSP comparison a SPS.
            \item The quality of solutions found by GSP is comparable with the 
            quality of solutions found by SPS.
            \item Coalition Formation problem can approximate the results 
            of a set partion problem in {\bf less} time complexity.
        \end{itemize}
        We are focused on a {\bf centralized coordinator} in the future works we want to 
        perform a {\bf distributed coordiantion}.

        That strategy should be more {\bf flessible}, {\bf adaptive} at the situation on the traveling
        orders then {\bf fault-tolenace}. 
    \end{frame}









\chapter{Background and Related Works}\label{chap:related}
In this section, we detail the main issues for \mrs coordination 
in industrial domains, then we provide a detailed discussion on coordination
approaches, highlighting challenges and main solution techniques.

\section{Multi-Robot System for Logistics Applications}
In this thesis we focus on industrial scenarios where robots have a high
degree of autonomy and operate in a dynamic environment.
\\
In article \cite{maxsum} robots must cooperate to maximizes the number of packages
completed in the unit of time. To this end a crucial component is to avoid interferences
when moving in the enviroment. They have formalised that problem as a Distributed 
Constrained Optimization problem based their solution on the binary max-sum algorithm.
\\
The most recent paper in logistic scenario \cite{mapd}, uses a Token Passing approach 
that solves the pickup-and-delivery tasks. The resulting algorithm takes kinematic 
constraints during planning, computes continuos agent movements with given velocities
that work on non-holonomic robots.

In this work, we consider a similar setting where a set of robots are involved
in trasportations tasks for logistics. However, we focus on the specific problem
of task assignment.

% Sposta la sezione su MRS in logistics descrivendo almeno questi tre paper se ne trovi altri meglio: 

% Hang Ma, Jiaoyang Li, T. K. Satish Kumar, Sven Koenig:
% Lifelong Multi-Agent Path Finding for Online Pickup and Delivery Tasks. AAMAS 2017: 837-845

% Peter R. Wurman, Raffaello D'Andrea, and Mick Mountz. 2007. Coordinating hundreds of cooperative, autonomous vehicles in warehouses. In Proceedings of the 19th national conference on Innovative applications of artificial intelligence - Volume 2 (IAAI'07), William Cheetham (Ed.), Vol. 2. AAAI Press 1752-1759.


% A. Farinelli, N. Boscolo, E. Zanotto, E. Pagello, Advanced Approaches
% for Multi-Robot Coordination in Logistic Scenarios, April 2017.

% Per ogni paper devi spiegare bene quali sono le basi dell'approccio e come si differenzia dal tuo approccio (e.g. quello di koenig non considera la capacità etc.). Per ogni paper che descrivi devi considerare più o meno messa pagina o anche una non poche righe (solo per darti una idea del dettaglio con cui devi descrivere i paper).

\section{Coordination in Multi-Robot Systems}
Coordination for \mrs has been investigated from several diverse
perspectives and nowadays, there is a wide range of techniques that can be used to 
orchestrate the actions and movements of robots operationg in the same enviroment.
Specifically, the ability to effectively coordinate the actions of a \mrs is a key 
requirement in several applications domains that range from disaster response to 
environmental monitoring, militaty operations, manufacturing and logistics. 
In all such domains, coordination has been addressed using various frameworks and 
techniques and there are several survery papers dedicated to categorize such different
approaches and identifying most prominent issues when developing \mrs.
\\
In the paper \cite{cooros} present and evaluate new ROS package for coordinated 
multi-robot exploration. The packages allow completely distributed control and do 
not rely on (but allow) central controllers.
\\
Robotics systems may range from simple sensors, acquiring and processing data, to 
complex human-like machines, able to interact with the enviroment in fairly complex way. 
Moreover, it is not easy to give a definition of the level of autonomy that is required 
for a robot in order to be considered an entity acting in the enviroment, as opposed 
to a simple machine that provides services to the operator.
From an engineering standpoint,
the \mrs can improve the effectiveness of a robotic system either from the viewpoint 
of the performance in accomplishing
certain tasks, or in the robustness and reliability of the system,
which can be increased by modularization.
In fact, \mrs are useful not only when the robots can accomplishing different functions, but 
also when they have the same capabilities. Even when a single robot can achieve the given task,
the possibility of deploying a team of robots can improve the performance of the overall system.
Another significant development of \mrs is technological improvements both in the 
hardware and in the associated software are two of the key reasons beyond the growing
interest in \mrs. The increased availability of complex sensor devices and robotic platforms
in the research laboratories favored their development and customization, resulting
in robots equipped with reliable and effective hardware that improves their basic 
capabilities. In addiction, the software techniques developed for the robotic applications
take advantage of the hardware improvements and provide complex and reliable solutions for the basic 
tasks that a robot should be able to perform, while acting in real world environments:
localization, path planning, object trasportation, object recognition and tracking, etc. 
% approcci su auction 

% frase finale
Given our focus on logistic scenarios, here we restrict our attention to coordination
approaches based on optimization and specifically on task assignment as this the most 
common framework for my reference application domain.

% Quest parte va espansa molto descrivendo quali sono gli approcci per ottenere MRS coordination.

% Guarda questo paper per prendere spunto:

% Alessandro Farinelli, Luca Iocchi, Daniele Nardi:
% Multirobot systems: a classification focused on coordination. IEEE Trans. Systems, Man, and Cybernetics, Part B 34(5): 2015-2028 (2004)

% Descrivi in particolare gli approcci basati su auction (ad esempio questo:

% Robert Zlot, Anthony Stentz:
% Market-based Multirobot Coordination for Complex Tasks. I. J. Robotics Res. 25(1): 73-101 (2006)
%  )




\chapter{Problem formalization and Solution}\label{chap:problem}
In this section we detail our reference scenario for MRS coordination and formalization problem.

\section{Description}
Our reference scenario is based on a warehouse that stores items of various types.
Such items must be composed together to satisfy orders that arrive based on customers’ demand.
The items of various types are stored in particular section of the building (\textit{loading bay})
and must be trasported to a set of \textit{unloading bays} where such items are then 
packed together by human operators. The set of items to be trasported and where they should
go depends on the orders.

In our domain a set of robots is responsible for the transporting items from
the loading bays to the unloading bays and the system goal is to maximize the
throughput of the orders, i.e., to maximize the number of orders completed in
the unit of time. Now, robots involved in transportation tasks move around
the warehouse and are likely to interfere when they move in close proximity,
and this can become a major source of inefficiency (e.g., robots must slow down
and they might even collide causing serious delays in the system).

Hence, a crucial aspect to maintain highly efficient and safe operations is to minimize the
possible spatial interferences between robots. Specifically, here we propose to
take this interferences into account in the task assignment process and assign
tasks to robots so to reduce the possible interferences among the transportation
robots.


\section{Fomalization}
In this section we formalize the MRS coordination problem described above as a task allocation problem
where the robots must be allocated to transportation tasks. 

In our formalization if transportation tasks are more than the available robots at each time step only a subset 
of tasks will be allocated. However, since the task allocation process is repeated over time robots 
effectively serve a sequence of tasks. 

In more detail, my model considers a set of items of different types $E = \{ e_1,...,e_N\}$,
stored in a specific loading bay ($L$). The warehouse must serve a set of orders 
$O=\{o_1,...,o_M\}$. Orders are processed in one or more than one of the unloading bays ($U_i$).
Each order is defined by a vector of demand for each item type (the number of required 
items to close the order). Hence, $o_j = < d_{1,j},...,d_{N,j}>$, where $d_{i,j}$ is the 
demand for order $j$ of items of type $i$. When an order is finished a new one arrives,
and we assume to have no knowledge on future orders.
The orders induce a set of $N \times M$ trasportation tasks $T = {t_{i,j}}$, with 
$t_{i,j} = < d_{i,j}, dst_{i,j}, P_{i,j}>$, where $t_{i,j}$ defines the task of transporting 
$d_{i,j}$ items of type $i$ for order $o_j$ (hence to unloading bay $U_j$).
Each task has a destination bay for centralized coordination the $t_{i,j}$ has a set of edges
$P_{i,j}$ which respects the strategy used. 
I have a set of robot $R = \{r_1,...,r_K \}$ that can execute transportation tasks, where
each robot has a defined load capacity for each item type $C_k = <c_{1,k},...,c_{N,k}>$, 
hence $c_{i,k}$ is the load capacity of robot $k$ for items of type $i$.

We consider in our logistic scenario, homogenous robots, which have the same radius 
and the same capacity. Because often in the logistic environments robots are all 
equal. 



In this section I propose the approaches with centralized coordination.
The first strategy, mentioned above, is the CGS1:1 which consider only 
one task allocated for one robot.
The second strategy extends the first, the main concept of this strategy 
is merging the tasks for optimaze the capacity of robot.

\section{Cyclic Greedy Strategy Single robot : Single task (CGS1:1)}





\section{Cyclic Optimaze Strategy Single robot : Multiple task (COS1:N)}



% \chapter{Solution}\label{chap:solution}

\chapter{Empirical Setting} \label{chap:ros}
Writing software for robots is difficult, particularly as the scale and scope
of robotics continues to grow. Different types of robots can have wildly varying
hardware, making code reuse non trivial. On top of this, the magnitude of
the required code can be daunting, as it must contain a deep stack starting
from driver-level software and continuing up through perception, abstract
reasoning, and beyond.

\section{Robot Operating System (ROS)}
Our choice fell on ROS (Robot Operating System) which is a widespread
open-source, meta-operating system for a robot. It provides several services
that are commonly offered by an operating system, including hardware
abstraction, low-level device control, implementation of commonly-used functionality,
message-passing between processes, and package management. It is
worth noting that the full source code of ROS is publicly available, ROS is
distributed under the terms of the BSD license, which allows the development
of both non-commercial and commercial projects.

\subsection{Nomencalture and Architecture}
In this section we simply outline the terminology adopted in the ROS
community to allow an easy comprehension of the following discussion.
\\
The fundamental concepts of the ROS implementation are \textit{nodes}, \textit{messages},
\textit{topics}, and \textit{services}.
In ROS a system is typically comprised of many nodes. In this context, the term
\textit{"node"} is interchangeable with \textit{"software module"}. The use of term 
\textit{"node"} arises from visualization of ROS-based systems at runtime:
when many nodes are running, it is convenient to render the peer-to-peer communications
as a graph, called the \textit{computation graph}, with process as graph nodes and 
the peer-to-peer links as arcs.
\\
Nodes communicate with each other by passing \textit{messages}. A message is a
a strictly typed data structure. Standard primitive types (integer, floating
point, boolean, etc.) are supported, as are arrays of primitive types and
constants. Messages can be composed of other messages, and arrays of other
messages, nested arbitrarily deep. Messages descriptions are usually stored
in \texttt{my\_package/msg/MyMessageType.msg} and define the data structures for
messages sent in ROS, called custom message.
\\ 
Here is a simple example of a \texttt{*.msg} file that uses a header, some integer primitive,
arrays of integer and array of other \texttt{*.msg} files. The message is specified 
in a language neutral interface definition language (IDL) which uses very short text 
files to describe its fields and allow an easy composition of complex messages:
\begin{multicols}{2}
\begin{lstlisting}
    Header      header
    bool        take
    bool        go_home
    uint32      ID_ROBOT
    uint32      item
    uint32      order
    uint32      demand
    uint32      dst
    uint32      path_distance
    uint32[]    route
\end{lstlisting}
The custom message above rappresent a \texttt{Task.msg} which contains the basic
information to define a task in the system. Instead, the custom message below, rappresent 
a \texttt{Mission.msg} which is composed of task messages addressed to a specific robot.  
\begin{lstlisting}
    Header      header
    uint32      ID_ROBOT
    uint32      capacity
    Task[]      Mission
\end{lstlisting}
\end{multicols}

These simple high-level message definitions is then parsed and processed by a code 
generator module, one for each support language (currentrly C++), which generates 
native implementations that “feel” like native objects, and are automatically 
serialized and deserialized by ROS as messages are sent and received.
\\
A node sends a message by publishing it to a given \textit{topic}, which is simply
a string such as \texttt{/topic} or \texttt{/pkg/topic}. A node that is interested
in a certain kind of data will subscribe to the appropriate topic. There may be multiple
concurrent publishers and subscribers for a single topic, and a single node
may publish and/or subscribe to multiple topics. In general, publishers and
subscribers are not aware of each other existence (decoupling). 
% Figure TODO: report an example \textit{computation graph}
It is important to point out that because nodes connect to each other at runtime,
the graph can be \textit{dynamically} modified.
\\
Although the topic-based publish-subscribe model is a flexible communications
paradigm, its “broadcast” routing scheme is not appropriate forsynchronous transactions,
which can simplify the design of some nodes.
For this purpose ROS includes the concept of \textit{services}, defined by a string name
and a pair of strictly typed messages: one for the request and one for the
response. A providing node offers a service under a name and a client uses
the service by sending the request message and awaiting the reply.
\\
As for the topic-based paradigm a high-level description of a service is then
parsed and processed by a code generator module which generates the corresponding
native implementation in a supported target language.
Usually C++ messages are generated in \texttt{my\_package/msg\_gen/cpp/include/my\_package},
while C++ services are generated in \texttt{my\_package/srv\_gen/cpp/include/my\_package}.
% TODO figura di differenze tra topic e service
To support collaborative development, the ROS software system is organized into \textit{packages}.
A ROS package is simply a directory which contains an XML file describing the package
and stating any dependencies. A collection of ROS packages is a directory tree with ROS
packages at the leaves: a ROS package repository may thus contain an arbitrarily complex
scheme of subdirectories. This structure is primarily meant to partition the building of 
ROS-based software into small, manageable chunks of functionality.
\\
In ROS, a \textit{stack} of software is a cluster of nodes that does something
coherent as a whole, as is illustrated in the simple \textit{navigation} example reported
in Figure. %TODO 
To allow for “packaged” functionality such as a navigation system, ROS provides
a tool called \texttt{roslaunch}, which reads an XML-like description of a graph and instantiates
the graph on the cluster, optionally on specific hosts.
Thus ROS is able to instantiate a set of nodes with a single
command, once the nodes are described in a \texttt{launch} file, the simple usage
is:
\begin{lstlisting}
    roslaunch [package] [filename.launch]
\end{lstlisting}

\subsection{The Stage 2D Simulation}
For visualization purposes we adopted the Stage 2D robot simulator
which provides a virtual world populated by mobile robots and enriched with
sensors, actuators and both approximate and exact localization. Stage is
designed to be sufficiently simple to allow an easy set-up but at the same time
it is intended to be just realistic enough to enable users to move controllers
directly between Stage robots and real robots. 
\\
Stage is made available in ROS with the \texttt{stageros} node which wraps the 
simulator and exposes its functionality to the rest of the system. The following 
code reports how it is launched:

\begin{lstlisting}
<?xml version="1.0" encoding="UTF-8" ?>
<launch>
    <arg name="map" default="grid" />
    <arg name="stage_pkg" default="stage_ros"/>     <!-- stage_pkg:=stage for ROS Groovy -->
    <arg name="custom_stage" default="false" />
    <group unless="$(arg custom_stage)"> 
        <node name="stageros" pkg="$(arg stage_pkg)" type="stageros" 
        args="$(find patrolling_sim)/maps/$(arg map)/$(arg map).world" output="screen" />
    </group>
    <group if="$(arg custom_stage)">   
        <node name="stageros" pkg="$(arg stage_pkg)" type="stageros" 
        args="$(find patrolling_sim)/maps/$(arg map)/$(arg map).world" output="screen">
            <param name="base_frame" value="base_link" />
            <param name="laser_topic" value="base_scan" />
            <param name="laser_frame" value="base_laser_link" />
        </node> 
    </group>
</launch>
\end{lstlisting}

The \texttt{*.world} file specified tells Stage everything about the world,
from obstacles (usually represented via a \texttt{*.pgm} image), to robots and other 
objects. In particular, after the definition of some parameters related to general 
camera and GUI options, we specify the static map on which the robot has to navigate 
(we will describe its characteristics shortly) and finally we include two specific 
files which aims defining the properties of respectively the laser sensor and the robot.
The last instruction just throws the robot in the map by indicating it's $x$, $y$, $z$
and $\theta$ coordinates, this is summarized in:
\begin{lstlisting}
include "../hokuyo.inc"
include "../crobot.inc"
include "../floorplan.inc"
include "../cpoint.inc"
window
( size   [ 460 180 1 ]         
  rotate [ 0.000 0.000 ]    
  center [ 11.5 4.0 ]   
  scale 20
  show_data 1)
floorplan
( size [23.0 8.0  1] 
  pose [11.5 4.0 0 0]
  bitmap "model5.pgm")
include "robots.inc"
include "point.inc"
\end{lstlisting}
The first included file (hokuyo.inc) defines the physical and technical
properties of the particular laser range finders support that we adopt: we
define it to have a circular shape and to be mounted on top of the robot base
which has the same circular shape. As for the sensor properties we specify
the following parameters described in:
\begin{lstlisting}
define hokuyo ranger
(
  sensor(           
    range [ 0.0  5.0 ] # the max/min range reported by the scanner, in meters.
    fov 230            # the angular field of view of the scanner, in degrees.
    samples 1081       # the number of laser samples per scan.
    )
  # model properties
  color "orange"
  size [ 0.1 0.1 0.1 ]
  block( points 4
    point[0] [0 0]
    point[1] [0 1]
    point[2] [1 1]
    point[3] [1 0]
    z [0 1])
)
\end{lstlisting}
The second included file (crobot.inc) defines the physical properties of
the robot, as mentioned above we define it to have a circular shape which
is suffice for our purpose of having a mobile camera that moves around the world:
\begin{lstlisting}
    define crobot position(
    size [0.3 0.3 0.2]
    origin [0 0 0 0]
    gui_nose 0
    drive "diff"

    # This block approximates a circular shape of a Robot
    block( points 16
        point[0] [ 0.225 0.000 ]
        point[1] [ 0.208 0.086 ]
        point[2] [ 0.159 0.159 ]
        point[3] [ 0.086 0.208 ]
        point[4] [ 0.000 0.225 ]
        point[5] [ -0.086 0.208 ]
        point[6] [ -0.159 0.159 ]
        point[7] [ -0.208 0.086 ]
        point[8] [ -0.225 0.000 ]
        point[9] [ -0.208 -0.086 ]
        point[10] [ -0.159 -0.159 ]
        point[11] [ -0.086 -0.208 ]
        point[12] [ -0.000 -0.225 ]
        point[13] [ 0.086 -0.208 ]
        point[14] [ 0.159 -0.159 ]
        point[15] [ 0.208 -0.086 ]
        z [0 1]
    )
    
    hokuyo( pose [0.15 0 -0.1 0] )

    # Report error-free position in world coordinates
    localization "gps"
    #localization_origin [ 0 0 0 0 ]

    # Some more realistic localization error
    localization "odom"
    odom_error [ 0.01 0.01 0.0 0.1 ]
)
\end{lstlisting}
% TODO sistemare gli spazzi bianchi

\section{Localization and Navigation}

\subsection{Mapping}
\subsection{Localization}
\subsection{Navigation}
\subsection*{Trasformations}
\subsection*{Sensor Information}
\subsection*{Odometry Informations}
\subsection*{Base Controller}
\subsection{Global and Local planner algorithms}
\subsection*{Global planner}
\subsection*{Local planner}
\section{Cost-maps configurations}
\section{Recovery behaviours}



\chapter{Experiments}\label{chap:experiments}
\input{chapter/experiments.tex}


\begin{table}[hbt]
    \centering
    \begin{tabular}{|c|c|c|c|c|c|} \hline
    {\bf Configuration} & {\bf Algorithm} & {\bf $ \overline{Time}$} & {\bf $\overline{Interference}$} & {\bf $\overline{Distance}$} & {\bf $\bar{\sigma}(Distance)$}         \\ \hline
    *6/-/2               & \srst           & 218.32$[\pm 6.19]$        & 63.45   & 3747.90 &  87.8\\ \hline 
    *6/3/2               & \gsp            & 194.52$[\pm 6.42]$        & 49.65    & 3401.15 & 251.37  \\ 
    *                    & \sps            & 177.00$[\pm 1.99]$        & 49.34    & 3132.5 & 0  \\ \hline
    *6/5/2               & \gsp            & 142.08$[\pm 1.39]$       &   42.2        & 2714.25      &  206.43  \\
    *                    & \sps            & 138.98$[\pm 2.41]$       &  39.38   &  2601.25  &  156.47   \\ \hline
    *6/-/4               & \srst           & 124.52$[\pm 3.12]$        & 42    & 2194.75 &  114.2 \\ \hline
    *6/3/4               & \gsp            & 117.44$[\pm 1.85]$        & 35.75   & 1769 & 43.83  \\ 
    *                    & \sps            & 115.28$[\pm 4.10]$        & 33.5   & 1702.5 & 23.67  \\ \hline
    *6/5/4               & \gsp            &  93.4$[\pm 1.01]$         & 29     & 1688.5 &  34.5   \\
    *                     & \sps            &  91.8$[\pm 2.14]$        & 30.75  & 1546.5 & 35.8  \\ \hline
    *9/-/2               & \srst           & 292.24$[\pm 3.06]$      &  85.5  & 5201.5 & 34.76\\ \hline
    *9/3/2              & \gsp            & 265.72$[\pm 2.64]$      & 71.5   & 4491.5  & 0  \\ 
    *                   & \sps            & 240.74$[\pm 10.42]$        & 75.5    & 4232.5 & 310.43 \\ \hline
    * 9/5/2             & \gsp          & 232.84$[\pm 4.71]$      &  68.85  &  4041.25  &  236 \\
    *                   & \sps           &168.34$[\pm 2.03]$       &   50.5   & 3132.5 &  0  \\ \hline 
    *9/-/4               & \srst           & 178.55$[\pm 4.23]$     & 52  & 2755.75 & 135.8 \\ \hline
    *9/3/4               & \gsp            & 152.55$[\pm 2.87]$     & 46.75 & 2200 & 113.4 \\ 
    *                   & \sps            & 134.23$[\pm 3.25]$     &  40.63   & 2182.5 & 27 \\ \hline
    *9/5/4              & \gsp           &134.23$[\pm 3.26]$        & 40.6    & 2098.3     &   93.45    \\
    *                   & \sps           &93.05$[\pm 5.15]$         & 32.25    & 1530.25 &   0 \\ \hline    
    
    *21/-/2             & \srst           & 629.10$[\pm 8.84]$        & 154.6   &11773.5 &  229.75 \\ \hline
    @21/3/2              & \gsp            &       &   & & -  \\ 
    @21/5/2              & \gsp            &       &   & & -  \\ \hline
    *21/-/4              & \srst           & 402.12$[\pm 5.06]$     & 132.25  & 6232.35 &  295.1 \\ \hline
    @21/3/4              & \gsp            &  & &  & -  \\ 
    @21/5/4              & \gsp            &  & &  & -  \\ \hline
   *42/-/2              & \srst           & 1283.2$[\pm 25.27]$     & 336   & 22446.63 &  323.8 \\ \hline
    @42/3/2              & \gsp            & &  &  & -  \\ 
    @42/5/2              & \gsp            & &  &  & -  \\ \hline
    @42/-/4              & \srst           & &  &  &  - \\ \hline
    @42/3/4              & \gsp            & &  &  & -  \\ 
    @42/5/4              & \gsp            & &  &  & -  \\ \hline
\end{tabular}
\end{table}


































\chapter{Conclusions and Future Work}\label{chap:conclusions}
In this thesis we have shown how our system can perform pickup-and-delivery tasks.
We evaluate the performance of our system in a realistic simulation enviroment,
precisely on the ICE Laboratory build with ROS and stage. 
In particular we compared our task assignment approches with the baseline greedy 
approch.

We have discovered that a coalition formation problem can approximate the results 
of a set partion problem in less time complexity.

Because we are focused on a centralized coordinator in the future works we want to 
perform a distributed coordiantor using a recent stategy studied before to implement 
our approaches.
In more detail on \cite{mapd} they used a Token Passing strategy with kinematic 
constraints but consider only one task assigned to one robot at time.
We want to extend this limitations considering the possibility of assigning more 
tasks in one travel.
That strategy should be more flessible, adaptive at the situation on the traveling
orders then fault-tolenace. 

The scope set at the beginning of the thesis have been reached.
The results comfirmed the expectations. 

Then in coclusion the \gsp algorithm aproximate the \sps strategy on the same task set,
calculating the solution in polynomial time complexity.


% \chapter*{Acknowledgements}
% \addcontentsline{toc}{chapter}{\protect\numberline{}Acknowledgements}

% \chapter*{Bibliography}
\begin{thebibliography}{99}
    \bibitem{parker} L.E. Parker, 
    \emph{Distributed Intelligence: Overview of the Field and its
    Application in Multi-Robot Systems}, March 2008.

    \bibitem{maxsum} A. Farinelli, N. Boscolo, E. Zanotto, E. Pagello,
    \emph{Advanced Approaches for Multi-Robot Coordination in
    Logistic Scenarios}, April 2017.

    \bibitem{mapd} H. Ma, W. Hönig, T.K.S. Kumar, N. Ayanian, S. Koenig,
    \emph{Lifelong Path Planning with Kinematic Constraints
    for Multi-Agent Pickup and Delivery}, December 2018.

    \bibitem{cooros} T. Andre, D. Neuhold, C. Bettstetter,
    \emph{Coordinated Multi-Robot Exploration:
    Out of the Box Packages for ROS}, December 2014.

    \bibitem{cf_farinelli} A. Farinelli, M. Bicego, F. Bistaffa, S. D. Ramchurn,
    \emph{A Hierarchical Clustering Approach to Large-scale
    Near-optimal Coalition Formation with Quality
    Guarantees}, March 2017.

    \bibitem{cf_greedy} X. Zheng, S. Koenig,
    \emph{Greedy Approaches for Solving Task-Allocation Problems with Coalitions}
    , 2008.

    \bibitem{partition} M. Orlov,
    \emph{Efficient Generation of Set Partitions}, March 2002.

    \bibitem{farinelli_coo} A. Farinelli, L. Iocchi, D. Nardi,
    \emph{Multirobot Systems: A Classification
    Focused on Coordination}, October 2004.

    \bibitem{market-based} R. Zlot, A. Stentz, 
    \emph{Market-based Multirobot Coordination for Complex Tasks}, January 2006.

    \bibitem{mapf} H. MA, J. Li, T. K. S. Kumar, S. Koenig,
    \emph{Lifelong Multi-Agent Path Finding
    for Online Pickup and Delivery Tasks}, May 2017.

    \bibitem{decoo} R. Stranders, A. Farinelli, A. Rogers, N. R. Jennings,
    \emph{Decentralised Coordination of Mobile Sensors Using the Max-Sum Algorithm}, 2009.

    \bibitem{focoo} A. Farinelli, L. Iocchi, D. Nardi,
    \emph{Multirobot Systems: A Classification
    Focused on Coordination}, October 2004.

    \bibitem{autocoo}, C. Jones, M. J. Mataric,
    \emph{Automatic Synthesis of Communication-Based
    Coordinated Multi-Robot Systems}, October 2004.

    \bibitem{coocoo}, P. R. Wurman, R. D'AndreA, M. Mountz,
    \emph{Coordinating Hundreds of
    Cooperative, Autonomous
    Vehicles in Warehouses}, March 2008.
    
    
\end{thebibliography} 

% \bibliographystyle{splncs03}
% \bibliography{biblio}

\end{document}
