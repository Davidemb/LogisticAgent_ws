In this section, we detail the main issues for \mrs coordination 
in industrial domains, then we provide a detailed discussion on coordination
approaches, highlighting challenges and main solution techniques.

\section{Multi-Robot System for Logistics Applications}
In this thesis we focus on industrial scenarios where robots have a high
degree of autonomy and operate in a dynamic environment.
\\
In article \cite{maxsum} robots must cooperate to maximizes the number of packages
completed in the unit of time. To this end a crucial component is to avoid interferences
when moving in the enviroment. They have formalised that problem as a Distributed 
Constrained Optimization problem based their solution on the binary max-sum algorithm.
\\
The most recent paper in logistic scenario \cite{mapd}, uses a Token Passing approach 
that solves the pickup-and-delivery tasks. The resulting algorithm takes kinematic 
constraints during planning, computes continuos agent movements with given velocities
that work on non-holonomic robots.

In this work, we consider a similar setting where a set of robots are involved
in trasportations tasks for logistics. However, we focus on the specific problem
of task assignment.

% Sposta la sezione su MRS in logistics descrivendo almeno questi tre paper se ne trovi altri meglio: 

% Hang Ma, Jiaoyang Li, T. K. Satish Kumar, Sven Koenig:
% Lifelong Multi-Agent Path Finding for Online Pickup and Delivery Tasks. AAMAS 2017: 837-845

% Peter R. Wurman, Raffaello D'Andrea, and Mick Mountz. 2007. Coordinating hundreds of cooperative, autonomous vehicles in warehouses. In Proceedings of the 19th national conference on Innovative applications of artificial intelligence - Volume 2 (IAAI'07), William Cheetham (Ed.), Vol. 2. AAAI Press 1752-1759.


% A. Farinelli, N. Boscolo, E. Zanotto, E. Pagello, Advanced Approaches
% for Multi-Robot Coordination in Logistic Scenarios, April 2017.

% Per ogni paper devi spiegare bene quali sono le basi dell'approccio e come si differenzia dal tuo approccio (e.g. quello di koenig non considera la capacità etc.). Per ogni paper che descrivi devi considerare più o meno messa pagina o anche una non poche righe (solo per darti una idea del dettaglio con cui devi descrivere i paper).

\section{Coordination in Multi-Robot system}
Coordination for \mrs has been investigated from several diverse
perspectives and nowadays, there is a wide range of techniques that can be used to 
orchestrate the actions and movements of robots operationg in the same enviroment.
Specifically, the ability to effectively coordinate the actions of a \mrs is a key 
requirement in several applications domains that range from disaster response to 
environmental monitoring, militaty operations, manufacturing and logistics. 
In all such domains, coordination has been addressed using various frameworks and 
techniques and there are several survery papers dedicated to categorize such different
approaches and identifying most prominent issues when developing \mrs.
\\
In the paper \cite{cooros} present and evaluate new ROS package for coordinated 
multi-robot exploration. The packages allow completely distributed control and do 
not rely on (but allow) central controllers.
\\
Given our focus on logistic scenarios, here we restrict our attention to coordination
approaches based on optimization and specifically on task assignment as this the most 
common framework for my reference application domain.

% Quest parte va espansa molto descrivendo quali sono gli approcci per ottenere MRS coordination.

% Guarda questo paper per prendere spunto:

% Alessandro Farinelli, Luca Iocchi, Daniele Nardi:
% Multirobot systems: a classification focused on coordination. IEEE Trans. Systems, Man, and Cybernetics, Part B 34(5): 2015-2028 (2004)

% Descrivi in particolare gli approcci basati su auction (ad esempio questo:

% Robert Zlot, Anthony Stentz:
% Market-based Multirobot Coordination for Complex Tasks. I. J. Robotics Res. 25(1): 73-101 (2006)
%  )


