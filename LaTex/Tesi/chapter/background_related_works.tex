In this section, we detail the main issues for \mrs coordination 
in industrial domains, then we provide a detailed discussion on coordination
approaches, highlighting challenges and main solution techniques.

\section{Multi-Robot System for Industrial Applications}
In this thesis we also focus on industrial scenarios where robots have a high
degree of autonomy and operate in a dynamic environment.
\\
In article \cite{maxsum} robots must cooperate to maximizes the number of packages
completed in the unit of time. To this end a crucial component is to avoid interferences
when moving in the enviroment. They have formalised that problem as a Distributed 
Constrained Optimization problem based their solution on the binary max-sum algorithm.
\\
The most recent paper in logistic scenario \cite{mapd}, uses a Token Passing approach 
that solves the pickup-and-delivery tasks. The resulting algorithm takes kinematic 
constraints during planning, computes continuos agent movements with given velocities
that work on non-holonomic robots.

In this work, we consider a similar setting where a set of robots are involved
in trasportations tasks for logistics. However, we focus on the specific problem
of task assignment.

\section{Coordination in Multi-Robot system}
Coordination for \mrs has been investigated from several diverse
perspectives and nowadays, there is a wide range of techniques that can be used to 
orchestrate the actions and movements of robots operationg in the same enviroment.
Specifically, the ability to effectively coordinate the actions of a \mrs is a key 
requirement in several applications domains that range from disaster response to 
environmental monitoring, militaty operations, manufacturing and logistics. 
In all such domains, coordination has been addressed using various frameworks and 
techniques and there are several survery papers dedicated to categorize such different
approaches and identifying most prominent issues when developing \mrs.
\\
In the paper \cite{cooros} present and evaluate new ROS package for coordinated 
multi-robot exploration. The packages allow completely distributed control and do 
not rely on (but allow) central controllers.
\\
Given our focus on logistic scenarios, here we restrict our attention to coordination
approaches based on optimization and specifically on task assignment as this the most 
common framework for my reference application domain.