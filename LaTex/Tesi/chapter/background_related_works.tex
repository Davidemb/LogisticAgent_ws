In this section, we detail the main issues for Multi-Robot system coordination 
in industrial domains, then we provide a detailed discussion on coordination
approaches, highlighting challenges and main solution techniques.

\section{Multi-Robot system for Industrial Applications}
In this thesis I also focus on industrial scenarios where robots have a high
degree of autonomy and operate in a dynamic environment. 
% 
% 
% Qua c'e da menzionare i paper letti finora in generale
% 
% 
In this work, I consider a similar setting where a set or robots are involved
in trasportations tasks for logistics. However, I focus on the specific problem
of task assignment ...

% cosa faccio di diverso rispetto gli articoli in generale

\section{Coordination in Multi-Robot system}
Coordination for Multi-Robot system (MRS) has been investigated from several diverse
perspectives and nowadays, there is a wide range of techniques that can be used to 
orchestrate the actions and movements of robots operationg in the same enviroment.
Specifically, the ability to effectively coordinate the actions of a MRS is a key 
requirement in several applications domains that range from disaster response to 
environmental monitoring, militaty operations, manufacturing and logistics. 
In all such domains, coordination has been addressed using various frameworks and 
techniques and there are several survery papers dedicated to categorize such different
approaches and identifying most prominent issues when developing MRS.
% 
% 
% cosa fanno i paper letti fin'ora per coordinare i robot. ex: grafo, ecc
% 
% 
\\
Given my focus on logistic scenarios, here I restrict my attention to coordination
approaches based on optimization and specifically on task assignment as this the most 
common framework for my reference application domain.

% cosa faccio io di diverso rispetto gli articoli. nello specifico della coordinazione

