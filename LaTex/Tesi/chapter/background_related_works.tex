In this section, we detail the main issues for \mrs coordination 
in industrial domains, then we provide a detailed discussion on coordination
approaches, highlighting challenges and main solution techniques.

\section{Multi-Robot System for Logistics Applications}
In this thesis we focus on industrial scenarios where robots have a high
degree of autonomy and operate in a dynamic environment.

% coocoo 
In this article \cite{coocoo} they presented the Kiva warehouse-management system 
create a paradigm for pick-pack-and-ship warehouse that improves worker productivity \footnote{https://www.amazonrobotics.com/\#/} .
The Kiva system uses movable storage shelves that can be lifted by small, autonomous robots.
By bringing the product to the worker, productivity is increased by a factor of two or more,
while simultaneously improving accountability and flexibility.
The key innovation in the Kiva system is the application of
inexpensive robots capable of lifting and carrying
three-foot-square shelving units, called \textit{inventory pods}.
The robots, called \textit{drive units}, transport the inventory pods from storage
locations to stations where workers can pick items off the shelves and
put them into shipping cartons. Throughout the
day, the picker stays in her station while a continuous stream of robots presents
pick-faces. By moving the inventory to the worker, rather than the
other way around, we typically see worker productivity at least double.
The Kiva drive units operate in a controlled, known enviroment, greatly simplifying
the design problem and making the solution pratical.
Another distinguishing attribute of \mrs is the extent to which agents are cooperative 
(in the sense that they must coordinate activities to achieve a system goal) 
or are self-interested and have independent, often conflicting, objectives.
Although the overall system is cooperative, the Kiva robots are essentially independent.
Warehouses and distribution centers play a critical role in the flow of goods from 
manufacturers to consumers. They serve as giant routing centers in which pallets
of products from different manufacturers are split, and the items are redirected
into outgoing containers.
The drive units are small enough to fit under the inventory pod and are outfitted
with a mechanical lifting mechanism that allows them to lift pods off
the ground. The pods consist of a stack of trays,
each of which is subdivided into bins. A variety of
tray sizes and bin sizes create the mixture of storage locations for the profile
of products the warehouse stores.
A Kiva installation is arranged on a
grid with storage zones in the middle and inventory stations spread around the perimeter.
The drive units are used to move the inventory pods with the correct bins
from their storage locations to the inventory stations where a pick worker removes 
the desired products from the desired bin. Note that the pod
has four faces, and the drive unit may need to rotate the pod in order to present
the correct face.
When a picker is done with a pod, the drive unit
stores it in an empty storage location.
For compute the path planning the environment is defined such as a grind costitutes 
a two-dimensional graph of paths that may be given weights at design time. 
The drive units use a standard implementation of $A^*$ to plan paths to storage locations 
and inventory stations. The drive unit agents also maintain a list of highlevel 
goals and are responsible for prioritizing the goals and accomplishing them as efficiently
as possible. Then the drive unit agent decides which station to visit first and in 
what sequence to show the faces to minimize travel time. 

% token passing

The most recent papers in logistic scenario \cite{mapf} and \cite{mapd}, are definited 
the Multi-Agent Pickup and Delivery problem (MAPD) where a large number of agents attend to
stream of incoming pickup-and-delivery tasks. They using a Token Passing (TP) approach
for implement MAPD algorithm that is efficient and effective. The MAPD algorithm takes 
kinematic constraints of real robot into account direclty during planning, computes 
continuous agent movements with given velocities that work on non-holonomic robot rather than
discrete agent movements with uniform velocity.
TP assumes, like many Multi-Agent pathfinding algorithms, discrete agent movements 
in the main compass directions with uniform velocity but can use a post-processing step 
to adapt its paths to continuous forward movements with given translational velocities 
and point turns with given rotational velocities. 
Unfortunately, the resulting paths might then not be effective since planning is oblivious 
to this transformation. TP needs to repeatdly plan time-minimal paths for agents 
that avoid collisions with the paths of the other agents. 
The important implication for this paper is that agents repeatdly plan paths for themselves,
considering the other agents as dynamic obstacles that follow their paths and with which 
collisions need to be avoided. The agents use space-time $A^*$ for this single-agent path planning.
A set of endpoints is any subset of cells that contains at least all start cells of agents and all 
pickup and delivery cells of tasks. The pickup and delivery cells are called task endpoints.
The other endpoints are called non-task endpoints.
TP operates as follows for a given set of endpoints: It uses
a token (a synchronized block of shared memory) that stores
the task set and the current paths, one for each agent.
The system repeatedly updates the task set in the token to contain
all unassigned tasks in the system and then sends the token
to some agent that is currently not following a path. The
agent with the token considers all tasks in the task set whose
pickup and delivery cells are different from the end cells of
all paths in the token.

Overview of TP works:
\begin{enumerate}
    \item If such tasks exist, then the agent
    assigns itself that task among these tasks whose pickup cell
    it can arrive at the earliest, removes the task from the task
    set, computes two time-minimal paths in the token, one that
    moves the agent from its current cell to the pickup cell of
    the task and then one that moves the agent from the pickup
    cell to the delivery cell of the task, concatenates the two
    paths into one path, and stores the resulting path.

    \item If no such tasks exist and the agent is not in the delivery cell
    of any task in the task set, then it stores the empty path in
    the token (to wait at its current cell).

    \item Otherwise, the
    agent computes and stores a time-minimal path in the token
    that moves the agent from its current cell to some endpoint
    that is different from both the delivery cells of all tasks in
    the task set and from the end cells of all paths in the token.
\end{enumerate}
Each path the
agent computes has two properties: (1) It avoids collisions
with all other paths in the token; (2) No other paths in the
token use its end cell after its end time. Finally, the agent
releases the token, follows its path, and waits at the end cell
of the path.
Tey demonstrate the benefit of their approach for automated warehouse.
Otherwise, they take kinematic constraints for instance the robot can turn around only 90 degree at time
and can perform only one task at time \footnote{Lifelong Path Planning with Kinematic Constraints for Multi-Agent Pickup and Delivery https://www.youtube.com/watch?v=RTJvJYJVxJk\&t=30s}. 
This limitation in our system does not preside.

% Profe logic 
In article \cite{maxsum} they focus on approaches that are based on algorithms
widely used to solve graphical models and constraint optimization problems, such as
the max-sum algorithm. They analyse the coordination problem faced by a set of robots
operating in a warehouse logistic application. In this context robots must trasport 
items from loading to unloading bays so to complete packages to be delivered to customers.
Robots must cooperate to maximizes the number of packages completed in the unit of time.
To this end crucial component is to avoid interferences when moving in the enviroment. 
They show how such problem can be formalised as a Distributed Constrained Optimization 
problem (DCOP )and they provide a solution based on the binary max-sum algorithm.
In more detail, in this paper \cite{maxsum}, they provide DCOP model for the task 
assignment problem faced by robots involved in logistics operations in a warehouse.
Among the various solution approaches for DCOPs they advocate the use of
heuristic algorithms, and specifically the max-sum, an iterative message passing
approach that has been shown to provide solutions of high quality for systems
operating in real-time and with limited computation and communication resources.
When the decicions of one robot affect only a small subset of team, because
the message update step, a key operation of max-sum, has a computation completely that 
is exponential in the number of robots that can perform the same task. 
This exponential element can be a significant limitation for large scale, real-time systems.
To combat this, they show that for specifc types of constraints and using binary variables,
such exponential element can be reduced to a polynomial. Hence, they can use the max-sum 
approach for large-scale systems that must operate with real-time constraints.


In this work, we consider a similar setting where a set of robots are involved
in trasportations tasks for logistics. However, we focus on the specific problem
of task assignment.

\section{Coordination in Multi-Robot Systems}
Coordination for \mrs has been investigated from several diverse
perspectives and nowadays, there is a wide range of techniques that can be used to 
orchestrate the actions and movements of robots operationg in the same enviroment.
Specifically, the ability to effectively coordinate the actions of a \mrs is a key 
requirement in several applications domains that range from disaster response to 
environmental monitoring, militaty operations, manufacturing and logistics. 
In all such domains, coordination has been addressed using various frameworks and 
techniques and there are several survery papers dedicated to categorize such different
approaches and identifying most prominent issues when developing \mrs.

In this paper \cite{cooros} present and evaluate new ROS package for coordinated 
multi-robot exploration. The packages allow completely distributed control and do 
not rely on (but allow) central controllers.
Their integration including application layer
protocols allows out of the box installation and execution. The
communication package enables reliable ad hoc communication
allowing to exchange local maps between robots which are
merged to a global map (for more detail see section \ref{ros:glplanner}).
Exploration uses the global map
to spatially spread robots and decrease exploration time. The
intention of the implementation is to offer basic functionality for
coordinated multi-robot systems and to enable other research
groups to experimentally work on multi-robot systems.
They use the terms "loacl" and "global" to distinguish contexts of a single robot and
the complete multi-robot system. 
A local map is the map created by each individual robot, while the global map includes
local maps of all robots. Communication enables the exchange of data. Lastly, coordinated
exploration utilizes communication and the global map to organize the \mrs by assigning
frontiers to robots. 

They contribution in to present ROS package that enable Multi-Robot exploration implementing 
the aforementioned required components:
\begin{itemize}
    \item ad hoc communication between robots.
    \item construction of global maps from local maps.
    \item exploration of unknown environments.
\end{itemize}

% paper profe 

In this work \cite{focoo} they focus on \mrs coordination, presented a survery of recent 
work in the area by specifically examining the forms of cooperation and coordination realized 
in the \mrs.
Robotics systems may range from simple sensors, acquiring and processing data, to 
complex human-like machines, able to interact with the enviroment in fairly complex way. 
Moreover, it is not easy to give a definition of the level of autonomy that is required 
for a robot in order to be considered an entity acting in the enviroment, as opposed 
to a simple machine that provides services to the operator.
From an engineering standpoint,
the \mrs can improve the effectiveness of a robotic system either from the viewpoint 
of the performance in accomplishing
certain tasks, or in the robustness and reliability of the system,
which can be increased by modularization.
In fact, \mrs are useful not only when the robots can accomplishing different functions, but 
also when they have the same capabilities. Even when a single robot can achieve the given task,
the possibility of deploying a team of robots can improve the performance of the overall system.
Another significant development of \mrs is technological improvements both in the 
hardware and in the associated software are two of the key reasons beyond the growing
interest in \mrs. The increased availability of complex sensor devices and robotic platforms
in the research laboratories favored their development and customization, resulting
in robots equipped with reliable and effective hardware that improves their basic 
capabilities. In addiction, the software techniques developed for the robotic applications
take advantage of the hardware improvements and provide complex and reliable solutions for the basic 
tasks that a robot should be able to perform, while acting in real world environments:
localization, path planning, object trasportation, object recognition and tracking, etc. 
In addiction, the work in this area can be classified from several points of view. 
Their main motivation is the study and evaluation of the ability to take advantage
of coordination to improve system performance. Therefore,
the classification we propose is focused on the coordination aspects 
and thus inspired by the relationships with the field of multi-agent systems.
They proposed the taxonomy for classifying the works on \mrs in characterized by two 
groups of dimensions: \textit{coordinatin dimensions} and \textit{system dimensions}.
For a suitable classification of the works it is important to clearly define the dimensions
that are used:
\begin{itemize}
    \item \textit{cooperation level:} is the ability of the system to cooperate in order 
    to accomplish a specific task. A cooperative system is composed of "robots that operate together
    to perform some global task". 
    \item \textit{knowledge level:} is concerned with the knowledge that each robot in the team
    has about its team mates.
    \item \textit{coordinatin level:} is the mechanisms used for cooperation in which the actions performed by each 
    robotic agents "in such a way that the whole ends up being a coherent and high-performance operaiton".
    The underlying feature is the \textit{coordiantion protcol}, that is defined as a set
    of rules that the robots must follow in order to interact with each other in the enviroment. 
    \item \textit{organization level:} is the way the decicion system is realized within the \mrs. 
    Introduces a distinction in the forms of coordination, distinguishing centralized approaches
    from distributed ones. In particular, a centralized system has a sgent (\textit{leader}) that is in
    charge of organizing the work of other agents;the leader
    is involved in the decision process for the whole team, while
    the other members can act only according to the directions of
    the leader. 
    Instead, a distributed system is composed ao agents which are completely autonomous in the 
    decision process with respect to each other; in this case of systems a leader does not exist. 
\end{itemize}
In this paper they have addressed the recent developments in the field of \mrs, focusing 
on those approaches that are targeted to specific applications and motivated by engineering
considerations. Specifically, they have presented a taxonomy with the aim of highlighting 
the coordination aspects of the recent proposals in the literature: we have defined a set of coordination
dimensions for the classification of the approaches to team coordination, together with a set of system dimensions that account
for the design choices that are more relevant to the team organization.

In out system using a central coordinator that give the actions to execute at all robots, which know the static environment and moving in their given task independently.

Another important article \cite{market-based} focus they work on coordinatin for complex tasks.
Complex task are tasks that can be solved in many possible way. Their work is currently 
limited to complex tasks that can be decomposed into multiple subtasks related by Boolen logic operators.
They generalizing task descriptions into task trees, which allows tasks to be traded
in a market setting at variable level of abstraction. 
The task allocation problem addresses the issue of finding task-to-robot assignments the optimize
global cost objectives. 
They address the general problem of allocating complex tasks to a team of autonomous robots.
Complex tasks are tasks that require high-level decision-making or planning,
and may have many potential solution strategies.
Complex tasks are usually identified with problems involving multiple interacting components.
These interactions can come from relationships between subtasks such
as Boolean logic operations or precedence constraints. 
Additionally, if there are multiple robots, complex tasks may
require reasoning about interactions between the robots executing them.
Specifically, they look at tasks hierarchically
related by \textit{AND} and \textit{OR} logical operators.
The main contribution of this paper are to identify the complex task allocation problem
can be more efficiently solved by not decoupling the solution into separate allocation
and decomposition phases, and to propose a solution concept that unifies these two stages.
The approach is not optimal, but produces highly efficient solutions in unknown
and dynamic domains using distributed local knowledge and decentralized planning 
to continually improve upon global costs.

In contrast to this article in our system we have simple tasks to compose in more complex tasks.
The interconnection between the various tasks is given by a heuristic that normalizes the cost of the route 
based on the number of objects transported in a composed task.

% frase finale
Given our focus on logistic scenarios, here we restrict our attention to coordination
approaches based on optimization and specifically on task assignment as this the most 
common framework for my reference application domain.
