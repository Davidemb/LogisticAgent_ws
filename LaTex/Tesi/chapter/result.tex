In this chapter we will present the results of the empirical experiments conducted 
to test the three strategies in a simulated autonomous warehouse with a central 
coordinator.

The experiments were performed on Intel Core i5-2520M CPU 2.50GHz $\times$ 4
, ram 7,7 GiB \footnote{For more about architecture used https://support.lenovo.com/my/it/solutions/pd015734}.

We were made 10 executions for each configuration.
The configuration of our experiments are:
\begin{itemize}
    \item the size of the task set $\mathcal{T}$, the fisrt number reported on configuration column.
    \item the capacities $C$ of the robots team. Note that in the basic method (\srst) we do not consider the capacity of robots because a task is assigned to each robot regardless of the demand of the tasks.
    \item the last number reported on configuration column is the number of the team size.
\end{itemize}

All of this experiments is performed on ICE laboraty map \footnote{See on site the Computer Engineering for Industry 4.0 http://www.di.univr.it/}. 

The second column of the table \ref{tab:result} reported the algorithm used to that experiment.
For see the different results we sorted the algorithms for the average time to notice the improving performance aspects. 

The next column reported the average time of the experiments and the standard error of the mean (SEM)\footnote{See https://en.wikipedia.org/wiki/Standard\_error}. 

The study sample can be described entirely by two parameters: 
the mean and the standard deviation ($\sigma$). The $\sigma$ represents the variability within the sample;
the larger the $\sigma$, the higher the variability within the sample. 
Although it is clear that samples should always be summarized by the mean and $\sigma$. 
The $\bar{\sigma}$ is used in inferential statistics to give an estimate of how the mean of the sample is related to the mean of the underlying population.
Although the $\sigma$ and the $\bar{\sigma}$ are related, they give two very different types of information.
Whereas the $\sigma$ estimates the variability in the study sample, 
the $\bar{\sigma}$ estimates the precision and uncertainty of how the study sample represents the underlying population.
In other words, the $\sigma$ tells us the distribution of individual data points around the mean, and the $\bar{\sigma}$ informs us how precise our estimate of the mean is.

The formula for the sample standard deviation is:
\[ \sigma={\sqrt {\frac {\sum _{i=1}^{N}(x_{i}-{\bar {x}})^{2}}{N-1}}} \]
where $\{x_1, \cdots, x_N\}$ are the observed values of the sample items, $\bar{x}$ it the mean value of these observations, 
and $N$ is the number of observations in the sample. 

The standar error of the mean ($\bar{\sigma}$) can be expressed as:
\[\bar{\sigma} ={\frac {\sigma }{\sqrt {N}}}\]
where $\sigma$ is the standar deviation of the population and $N$ is the size (number of observations) of the sample.

In the next columns are reported the average time of the simulation  ($\overline{Time}$), tha average interference ($\overline{Interference}$) and the average distance traveled ($\overline{Distance}$).
In the last column are reported the average $\bar{\sigma}$ for the team robots. In other worlds for one experiment are 
calculated the $\bar{\sigma}$ of the distance for every robot and after is calculated the average.   

\begin{table}[hbt]
    \centering
    \begin{tabular}{|c|c|c|c|c|c|} \hline
    {\bf Configuration} & {\bf Algorithm} & {\bf $ \overline{Time}$} & {\bf $\overline{Interference}$} & {\bf $\overline{Distance}$} & {\bf $\bar{\sigma}(Distance)$}         \\ \hline
    6/-/2               & \srst           & 218.32$[\pm 6.19]$        & 63.45   & 3747.90 &  87.8\\ \hline 
    6/3/2               & \gsp            & 194.52$[\pm 6.42]$        & 49.65    & 3401.15 & 251.37  \\ 
                        & \sps            & 177.00$[\pm 1.99]$        & 49.34    & 3132.5 & 0  \\ \hline
    6/5/2               & \gsp            & 142.08$[\pm 1.39]$       &   42.2        & 2714.25      &  206.43  \\
                        & \sps            & 138.98$[\pm 2.41]$       &  39.38   &  2601.25  &  156.47   \\ \hline
    6/-/4               & \srst           & 124.52$[\pm 3.12]$        & 42    & 2194.75 &  114.2 \\ \hline
    6/3/4               & \gsp            & 117.44$[\pm 1.85]$        & 35.75   & 1769 & 43.83  \\ 
                        & \sps            & 115.28$[\pm 4.10]$        & 33.5   & 1702.5 & 23.67  \\ \hline
    6/5/4               & \gsp            &  93.4$[\pm 1.01]$         & 29     & 1688.5 &  34.5   \\
                         & \sps            &  91.8$[\pm 2.14]$        & 30.75  & 1546.5 & 35.8  \\ \hline
    9/-/2               & \srst           & 292.24$[\pm 3.06]$      &  85.5  & 5201.5 & 34.76\\ \hline
    9/3/2              & \gsp            & 265.72$[\pm 2.64]$      & 71.5   & 4491.5  & 0  \\ 
                       & \sps            & 240.74$[\pm 10.42]$        & 75.5    & 4232.5 & 310.43 \\ \hline
     9/5/2             & \gsp          & 232.84$[\pm 4.71]$      &  68.85  &  4041.25  &  236 \\
                       & \sps           &168.34$[\pm 2.03]$       &   50.5   & 3132.5 &  0  \\ \hline 
    9/-/4               & \srst           & 178.55$[\pm 4.23]$     & 52  & 2755.75 & 135.8 \\ \hline
    9/3/4               & \gsp            & 152.55$[\pm 2.87]$     & 46.75 & 2200 & 113.4 \\ 
                       & \sps            & 134.23$[\pm 3.25]$     &  40.63   & 2182.5 & 27 \\ \hline
    9/5/4              & \gsp           &134.23$[\pm 3.26]$        & 40.6    & 2098.3     &   93.45    \\
                       & \sps           &93.05$[\pm 5.15]$         & 32.25    & 1530.25 &   0 \\ \hline    
    21/-/2             & \srst           & 629.10$[\pm 8.84]$        & 154.6   &11773.5 &  229.75 \\ \hline
    21/3/2              & \gsp            & 561.93$[\pm 8.00]$     &  134.3  &10133.16 &  201.2   \\ 
    21/5/2              & \gsp            & 497.45$[\pm 6.15]$      & 126  & 9079 & 210.4  \\ \hline
    21/-/4              & \srst           & 402.12$[\pm 5.06]$     & 132.25  & 6232.35 &  295.1 \\ \hline
    21/3/4              & \gsp            & 343.23$[\pm 6.10]$ & 98.23 & 5231.25 & 342.2   \\ 
    21/5/4              & \gsp            & 294.40$[\pm 7.60]$ & 77.63 & 4683.25 & 367.5  \\ \hline
\end{tabular}
\caption{Results of our experiments on ICE Laboratory}
\label{tab:result}
\end{table}

% consideraioni dei risultai
\newpage

From the table \ref{tab:result} shown we can notice, as previously expected 
considerations, that the strategy \gsp approximates the \sps strategy. 
Based on the complexity of the strategies used, the results respects inital expectations. 
Because using a polynomial complexity it is possible to have a solution close to the 
calculated solution for a more complex and complete problem. In other words, 
by calculating all the possible solutions we can have a good task allocation spending 
less computational time. 

In the last column ($\bar{\sigma}(Distance)$) we can see that the task allocation 
is  not always the same. It means that in different experiments, which the same configuration,
are assigned different tasks for different robots. 
If the value of the $\bar{\sigma}$ is equal to zero then it means that the task allocation is 
always the same for 10 times, that define a tightly allocation. In contrast, higher value of $\bar{\sigma}$
define a loosely allocation.  
























