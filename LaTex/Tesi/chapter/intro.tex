One of the fundamental areas in Robotics is multi-robot systems. More particularly,
this thesis addresses the cooperation of a team of mobile robots in logistic missions.
the main aspects studied herein are strategies for effective logistic performance,
agent's coordination, scalability and applicability in real-life situations.
% Real-life situation?
\newline
This introductory chapter presents the context of the research in order to clarify
the motivation and significance of the problem. 
In addition, some guidelines about Multi-Robot systems in general and, more specifically,
agents in logistic missions are herein introduced to lay the groundwork to 
approach the problem in hands. 
Finally, an overview of the document is given. 

\section{Context and Motivation}
In recent years, robotics has been one of the scientific fields with the most substantial
advances. Within the diverse areas that it embraces, mobile robotics has
had great focus in the last decades from roboticists (i.e., researchers on robotics)
around the world. In particular, issues like autonomous navigation, path planning,
self-localization, coordination of robots, cooperative dynamics, mapping, exploration 
and coverage have become popular and have benefited from the progress
of artificial intelligence, control theory, real-time systems, sensors’ development,
electronics, communication systems and systems integration [Parker, 2008].
\newline
Nowadays, we expect to see robots with many different shapes operating in
different environments as on land, underwater, in the air, suspended on wires,
climbing and so on. This evident growth is extremely motivating for the development
 and contribution of new developments by the community.
\newline
Security applications are a fundamental task with unquestionable impact on
society. Combining this fact with the technological evolution observed in the last
decades, it becomes clear that robot assistance can be a valuable resource by
taking advantage of robots’ expendability. In particular, multi-robot allocator task for logistic applications
has high utility and is considered as a contemporary area with some relevant work
presented in the last decade, especially in terms of strategies for coordinating
teams of robots. However, many of the studies in the literature present unrealistic
simplifications, strong limitations or questionable applicability as illustrated later
on. Therefore, there is an eminent potential to explore in this context.
\newline
Moreover, the allocator task for logistic applications problem is very challenging
in the context of Multi-Robot systems, because agents must navigate autonomously,
coordinate their actions in a distributed or centralized way and acquire information
about the surrounding space, possibly with communication constraints and independently
of the number of robots in the team and the enviroment's dimension.
All of these features lead to an excellent case study in mobile robotics
and conclusions drawn from such studies may support the development of future
approaches not only in the logistic domain but also in multi-robot systems, in
general.

\section{Multi-Robot Systems}
In many applications, an autonomous mobile robot equipped with different
sensors may adequately complete a given assignment. However, in several situations,
it proves to be more expensive, less efficient and less robust than using a multi-robot system.
In some cases, due to the need of combining different tasks and the dynamics of the environment.

Some characteristics of multi-robot systems include distributed control, autonomy,
communicative agents and greater fault-tolerance.
A single robot may be vulnerable to hostile environments or attackers, for example, in military actions.
In such scenarios, agents would greatly benefit from the assistance of nearby agents during emergencies,
failures or malfunctions.

One of the main difficulties when approaching these systems is to coordinate
many robots to perform a complex, global task in an efficient manner, maximizing
group performance under a wide range of conditions, with the flexibility to take
advantage of the resources available, embrace the requirements and constraints
imposed and resolve issues like action selection, coherence, conflict resolution and
communication. This cannot be done by just increasing the number of robots
assigned to a task.
A coordination mechanism must exist to establish relationships between agents so
that they can accomplish the mission effectively.

\section{The Multi-Robot system for logistic applications}
Logistic apllication an infrastructure with multiple robots is no different than other multi-robot
assignments, in the sense that it incorporates all the previously mentioned
characteristics of Multi-Robot system. To understand this problem, it is important to firstly
introduce the definition of logistic application.

\begin{mydef}
    Industrial Logistics, the set of operations related to the procurement,
    destination and storage of materials and products of large industry; the coordination and provisioning of people or things for the purpose of higher production efficiency.
\end{mydef}







