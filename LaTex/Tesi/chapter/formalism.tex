In this section we detail our reference scenario for MRS coordination and as well 
as the formalization of the problem we addressed and the solutions we propose.

\section{Description}
Our reference scenario is based on a warehouse that stores items of various types.
Such items must be composed together to satisfy orders that arrive based on customers’ demand.
The items of various types are stored in particular sections of the building (\textit{loading bay})
and must be trasported to a set of \textit{unloading bays} where such items are then 
packed together by human operators. The set of items to be trasported and where they should
go depends on the orders.

In our domain a set of robots is responsible for transporting items from
the loading bays to the unloading bays and the system goal is to maximize the
throughput of the orders, i.e., to maximize the number of orders completed in
the unit of time. Now, robots involved in transportation tasks move around
the warehouse and are likely to interfere when they move in close proximity,
and this can become a major source of inefficiency (e.g., robots must slow down
and they might even collide causing serious delays in the system).

Hence, a crucial aspect to maintain highly efficient and safe operations is to minimize the
possible spatial interferences between robots.
%  dire che ci focaliziamo sull'unione di pezzi del path


\section{Fomalization}
In this section we formalize the MRS coordination problem described above as a task allocation problem
where the robots must be allocated to transportation tasks. 

In our formalization we have a finite set of task. A robot can execute a task if it is available else the robot will go at start position.
For how the system is built, a robot to request a task must arrive at the previous vertex of the loading bay.

In more detail, our model considers a set of items of different types $E = \{ e_1,...,e_N\}$,
stored in a specific loading bay ($L$). The warehouse must serve a set of orders 
$O=\{o_1,...,o_M\}$. Orders are processed in one or more than one of the unloading bays ($U_i$).
Each order is defined by a vector of demand for each item type (the number of required 
items to close the order). Hence, $o_j = < d_{1,j},...,d_{N,j}>$, where $d_{i,j}$ is the 
demand for order $j$ of items of type $i$. When an order is finished a new one arrives,
until the set of task is finished.
The orders induce a set of $N \times M$ trasportation tasks $T = {t_{i,j}}$, with 
$t_{i,j} = < d_{i,j}, dst_{i,j}, P_{i,j}>$, where $t_{i,j}$ defines the task of transporting 
$d_{i,j}$ items of type $i$ for order $o_j$ (hence to unloading bay $U_j$).
Each task has a destination bay for centralized coordination the $t_{i,j}$ has a set of edges
$P_{i,j}$ which respects the strategy used. 
We have a set of robot $R = \{r_1,...,r_K \}$ that can execute transportation tasks, where
each robot has a defined load capacity for each item type $C_k = <c_{1,k},...,c_{N,k}>$, 
hence $c_{i,k}$ is the load capacity of robot $k$ for items of type $i$.

We consider in our logistic scenario, homogenous robots, which have the same radius 
and the same capacity. Because often in the logistic environments robots are all 
equal. 


