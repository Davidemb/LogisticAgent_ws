Writing software for robots is difficult, particularly as the scale and scope
of robotics continues to grow. Different types of robots can have wildly varying
hardware, making code reuse non trivial. On top of this, the magnitude of
the required code can be daunting, as it must contain a deep stack starting
from driver-level software and continuing up through perception, abstract
reasoning, and beyond.

\section{Robot Operating System (ROS)}
Our choice fell on ROS (Robot Operating System) which is a widespread
open-source, meta-operating system for a robot. It provides several services
that are commonly offered by an operating system, including hardware
abstraction, low-level device control, implementation of commonly-used functionality,
message-passing between processes, and package management. It is
worth noting that the full source code of ROS is publicly available, ROS is
distributed under the terms of the BSD license, which allows the development
of both non-commercial and commercial projects.

\subsection{Nomencalture and Architecture}
In this section we simply outline the terminology adopted in the ROS
community to allow an easy comprehension of the following discussion.
\\
The fundamental concepts of the ROS implementation are \textit{nodes}, \textit{messages},
\textit{topics}, and \textit{services}.
In ROS a system is typically comprised of many nodes. In this context, the term
\textit{"node"} is interchangeable with \textit{"software module"}. The use of term 
\textit{"node"} arises from visualization of ROS-based systems at runtime:
when many nodes are running, it is convenient to render the peer-to-peer communications
as a graph, called the \textit{computation graph}, with process as graph nodes and 
the peer-to-peer links as arcs.
\\
Nodes communicate with each other by passing \textit{messages}. A message is a
a strictly typed data structure. Standard primitive types (integer, floating
point, boolean, etc.) are supported, as are arrays of primitive types and
constants. Messages can be composed of other messages, and arrays of other
messages, nested arbitrarily deep. Messages descriptions are usually stored
in \texttt{my\_package/msg/MyMessageType.msg} and define the data structures for
messages sent in ROS, called custom message. 
Here is a simple example of a \texttt{*.msg} file that uses a header, some integer primitive,
arrays of integer and array of other \texttt{*.msg} files. The message is specified 
in a language neutral interface definition language (IDL) which uses very short text 
files to describe its fields and allow an easy composition of complex messages:
\\
\begin{multicols}{2}
\begin{lstlisting}
    Header      header
    bool        take
    bool        go_home
    uint32      ID_ROBOT
    uint32      item
    uint32      order
    uint32      demand
    uint32      dst
    uint32      path_distance
    uint32[]    route
\end{lstlisting}

\begin{lstlisting}
    Header      header
uint32      ID_ROBOT
uint32      capacity
Task[]      Mission
\end{lstlisting}

\end{multicols}




\section{The Stage 2D Simulation}

\section{Localization and Navigation}

\subsection{Mapping}
\subsection{Localization}
\subsection{Navigation}
\subsection*{Trasformations}
\subsection*{Sensor Information}
\subsection*{Odometry Informations}
\subsection*{Base Controller}
\subsection{Global and Local planner algorithms}
\subsection*{Global planner}
\subsection*{Local planner}
\section{Cost-maps configurations}
\section{Recovery behaviours}

