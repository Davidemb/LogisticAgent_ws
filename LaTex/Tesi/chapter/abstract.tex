\begin*{}
\newline
\newline
Robotics technology has recently matured sufficiently to deploy autonomous
robotic systems for daily use in several applications: from disaster response
to environmental monitoring and logistics.
In this project present and evaluate the principal difference of central and 
distributed allocator task coordinator. 
In these applications we address off-line coordination, by casting the Multi-Robot
logistics problem as a task assignment problem and proposing two solution 
techniques: Cyclic Greedy Strategy Single Robot Single Task (CGS1:1), which is 
a baseline greedy approach, and Cyclic Optimaze Strategy Single Robot Multiple Task 
(COS1:N), which is based on merging task for improve the spend time.
\\
And the last one is address on-line coordiantor, that is based on token passing (TP) approach.
We evaluate the performance of our system in a realistic simulation enviroment
(build with ROS and stage). In particular, in the simulated enviroment we compare
our task assignment approaches with previous off-line and on-line methods.
\newline
\newline
\textbf{Keywords:} Multi-Robot Task Allocator, logistic applications, Multi-Robot
systems, coordination, task assignment

% TODO: primi risultati e considierazioni

\end*{}


