\begin*{}
\newline
\newline
Robotics technology has recently matured sufficiently to deploy autonomous
robotic systems for daily use in several applications: from disaster response
to environmental monitoring and logistics.
In this project we present and evaluate the main difference of central allocator task coordinator. 
In this application we address off-line coordination, by casting the Multi-Robot
logistics problem as a task assignment problem and proposing three solution 
techniques: Single robot Single task (\srst), which is 
a baseline greedy approach,  Greedy Set Partition Strategy - Single robot Multiple task 
(\gsp) and Set Partition Strategy - Single robot Multiple task (\sps),
which are based on composing task to minimize the task completion time.
\\
We evaluate the performance of our system in a realistic simulation enviroment
(build with ROS and stage). In particular, in the simulated enviroment we compare
our task assignment approaches with the baseline greedy approch. Results show that fully
exploiting the capacities of robots is key to optimze system performance and that the 
\gsp achieves similar performance to \sps while being able to scale up to many tasks.
\newline
\newline
\textbf{Keywords:} Multi-Robot Task Allocation, logistic applications, Multi-Robot
systems, coordination, task assignment

% TODO: primi risultati e considierazioni

\end*{}


