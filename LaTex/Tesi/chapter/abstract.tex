\begin*{}
\newline
\newline
Robotics technology has recently matured sufficiently to deploy autonomous
robotic systems for daily use in several applications: from disaster response
to environmental monitoring and logistics.
In this project present and evaluate the principal difference of central and 
distributed allocator task coordinator. 
In these applications we address off-line coordination, by casting the Multi-Robot
logistics problem as a task assignment problem and proposing three solution 
techniques: Single robot Single task (\srst), which is 
a baseline greedy approach,  Greedy Set Partition Strategy Single robot Multiple task 
(\gsp) and Set Partition Strategy Single robot Multiple task (\sps),
which are based on merging task for improve the spend time.
\\
We evaluate the performance of our system in a realistic simulation enviroment
(build with ROS and stage). In particular, in the simulated enviroment we compare
our task assignment approaches with previously methods mentioned.
\newline
\newline
\textbf{Keywords:} Multi-Robot Task Allocator, logistic applications, Multi-Robot
systems, coordination, task assignment

% TODO: primi risultati e considierazioni

\end*{}


